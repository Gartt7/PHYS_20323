\documentclass[11pt]{article}
\usepackage{amsmath}
\usepackage{graphicx}
\usepackage{geometry}
\usepackage{xcolor}

\geometry{margin=1in}

\title{PHYS 20323/60323: Fall 2025 - LaTeX Example}
\author{}
\date{}

\begin{document}

\maketitle

\begin{enumerate}
    \item At time $t = 0$ a particle is represented by the wave function
    \begin{equation*}
    \Psi(x) = 
    \begin{cases}
    A\frac{x}{a}, & 0 \leq x \leq a \\
    A\frac{(b-x)}{(b-a)}, & a \leq x \leq b \\
    0, & \text{otherwise}
    \end{cases}
    \end{equation*}
    where $A$, $a$ and $b$ are constants.
    
    \begin{enumerate}
        \item (3.3 points) Normalize $\Psi$ (i.e., find $A$ terms of $a$ and $b$).
        \item (3.3 points) Where is the particle likely to be found at $t = 0$?.
        \item (3.4 points) What is the expectation value of $x$?.
    \end{enumerate}
    
    \item The following questions refer to stars in the Table below.
    
    Note: There may be multiple answers.
    
    \begin{center}
    \begin{tabular}{|l|c|c|c|c|c|c|}
    \hline
    Name & Mass & Luminosity & Lifetime & Temperature & Radius & Variable? \\
    \hline
    $\delta$ Scu. & 2.0 $M_{\odot}$ & & $5.0 \times 10^8$ years & & 2.0 $R_{\odot}$ & Y \\
    \hline
    $\gamma$ Del. & 0.7 $M_{\odot}$ & & $4.5 \times 10^{10}$ years & 5000 K & & N \\
    \hline
    $\beta$ Cyg. & 1.3 $M_{\odot}$ & 3.5 $L_{\odot}$ & & & & Y \\
    \hline
    $\eta$ Car. & 60. $M_{\odot}$ & $10^6$ $L_{\odot}$ & $8.0 \times 10^5$ years & & & Y \\
    \hline
    $\epsilon$ Eri. & 6.0 $M_{\odot}$ & $10^3$ $L_{\odot}$ & & 20,000 K & & N \\
    \hline
    $\alpha$ Cen. & 1.0 $M_{\odot}$ & & & 6000 K & 1.0 $R_{\odot}$ & N \\
    \hline
    \end{tabular}
    \end{center}
    
    \begin{enumerate}
        \item (4 points) Which of these stars will produce a planetary nebula.
        \item (4 points) Elements heavier than Carbon will be produced in which stars.
    \end{enumerate}
    
    \item An electron is found to be in the spin state (in the $z$-basis): $\chi = A \begin{pmatrix} 3i \\ 4 \end{pmatrix}$
    
    \begin{enumerate}
        \item (5 points) Determine the possible values of $A$ such that the state is normalized.
        \item (5 points) Find the expectation values of the operators $\textcolor{orange}{S_x}$, $\textcolor{red!80!black}{S_y}$, $\textcolor{orange!90!red}{S_z}$ and $\vec{S}^2$.
    \end{enumerate}
    
    The matrix representations in the $z$-basis for the components of electron spin operators are given by:
    \begin{equation*}
    \textcolor{orange}{S_x = \frac{\hbar}{2} \begin{pmatrix} 0 & 1 \\ 1 & 0 \end{pmatrix}}; \quad
    \textcolor{red!80!black}{S_y = \frac{\hbar}{2} \begin{pmatrix} 0 & -i \\ i & 0 \end{pmatrix}}; \quad
    \textcolor{orange!90!red}{S_z = \frac{\hbar}{2} \begin{pmatrix} 1 & 0 \\ 0 & -1 \end{pmatrix}}
    \end{equation*}
    
\end{enumerate}

\vfill
\begin{flushright}
Latex Example 3
\end{flushright}

\end{document}